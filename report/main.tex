%!TeX program = xelatex
\documentclass{article}
\usepackage{papers}

%%-------------------------------正文开始---------------------------%%
\begin{document}

%%-----------------------封面--------------------%%
\maketitle

%%------------------摘要-------------%%
\begin{abstract}
报告进行了 Firefox 在 RISC-V 平台,特别是在 Milk-V Pioneer Box 和 Sipeed LicheePi 4A 上的可用性测试。
测试涵盖了多个 Linux 发行版和 Firefox 版本,通过手动和自动测试评估其在 RISC-V 平台上的可用性。
手动测试验证了日常使用场景,自动测试利用 web-platform-tests 对 Firefox 进行了全面的兼容性检验。
结果显示,Firefox 能够在测试涉及的 RISC-V 设备上运行,但由于硬件性能和软件优化受限,存在无法使用硬件解码播放视频等问题。
总体而言,Firefox 在 RISC-V 平台上已经基本可用,但仍需进一步优化和支持。
\end{abstract}

\thispagestyle{empty} % 首页不显示页码

%%--------------------------目录页------------------------%%
\newpage
\tableofcontents

%%------------------------正文页从这里开始-------------------%
\newpage

\begin{markdown}

# 简介
由于
## 软件说明
Firefox 是一个免费和开源的浏览器,全世界 10% 的人使用 Firefox 作为他们的主要浏览器。Mozilla 是 Firefox 浏览器的开发商,主要提供专注于开放网络的产品。Firefox 是谷歌 Chrome 浏览器的替代品。

## 测试概述
本次测试在 RISC-V 设备 Milk-V Pioneer Box 和 Sipeed LicheePi 4A 的多个 Linux 发行版上对多个版本的 Firefox 进行了手动或自动的测试,对其目前在 RISC-V 上的可用性进行了较为全面的测试并得出了相应的结论。

测试过程中采用了大量使用的 web-platform-test 测试框架在部分发行版上对浏览器进行了测试,对其进行了较为全面且完整的验证。
在部分系统下还使用了 Speedometer3 和 Basemark 来测试浏览器浏览网页以及图形性能。

手动测试采用贴近真实用户日常使用场景的测试用例,模拟一般用户的操作方式对 Firefox 在测试环境上的日常浏览体验进行了评估。

# 环境说明

## 硬件环境
本次测试主要在 Milk-V Pioneer Box 和 Sipeed LicheePi 4A 上进行,机器硬件配置为:

Milk-V Pioneer Box:
- CPU: SG2042 64 Core C920@2.0GHz
- RAM: 4 channel 3200Hz 128GB DDR4 SODIMM (32GB * 4)
- SSD: PCIe 3.0 x 4 1TB
- GPU: AMD R5 230

Sipeed LicheePi 4A:
- CPU: TH1520, RISC-V 2.0G C910 x4
- RAM: 16 GB 64bit LPDDR4X-3733
- Storage: TF Card, 128 GB eMMC

## 软件环境

本次测试涵盖的系统版本和 Firefox 版本如下:
- Fedora 38(仅测试 Pioneer Box)
- - 软件仓库自带 Firefox(手动测试)
- - Nightly 版本(wpt 自动测试)
- openKylin 2.0
- - 软件仓库自带 Firefox(手动测试)
- - Nightly 版本(wpt 自动测试)
- RevyOS
- - 软件仓库自带 Firefox(手动测试)

其余未测试系统及原因如下:
- openEuler:截止测试时,仅有 CLI 版本
- openCloudOS:截止测试时,未找到可用版本
- Ubuntu:截止测试时,未找到可用版本
- Debian:RevyOS 可被视作 Debian
- Fedora:在 LicheePi 4A 上无法使用网口

## 环境搭建


### 安装系统

#### Sipeed LicheePi 4A

LicheePi 4A 各系统在 [支持矩阵](https://github.com/ruyisdk/support-matrix/tree/main/LicheePi4A) 上详细记载了安装过程,可作为参考。

- openKylin
从 [openKylin 官方](https://www.openkylin.top/downloads/index-cn.html) 下载对应镜像并解压后,使用 fastboot 刷入 emmc。

按住 boot 键后,上电/Reset 进入刷写模式。

uboot 分区镜像需要根据 RAM 大小选择版本:
```shell!
tar -xvf openKylin-Embedded-V2.0-Release-licheepi4a-riscv64.tar.xz
cd openKylin-Embedded-V2.0-Release-licheepi4a-riscv64/
sudo fastboot flash ram u-boot-with-spl-lpi4a(-16g).bin
sudo fastboot reboot
sudo fastboot flash uboot u-boot-with-spl-lpi4a(-16g).bin
sudo fastboot flash boot boot-lpi4a-20240720_171951.ext4
sudo fastboot flash root openKylin-2.0-licheepi4a-riscv64.img
```
默认账号密码为:`openkylin`:`openkylin`

- RevyOS
从 [ISCAS 镜像](https://mirror.iscas.ac.cn/revyos/extra/images/lpi4a/) 下载并解压:
```shell!
wget https://mirror.iscas.ac.cn/revyos/extra/images/lpi4a/20240720/boot-lpi4a-20240720_171951.ext4.zst
wget https://mirror.iscas.ac.cn/revyos/extra/images/lpi4a/20240720/u-boot-with-spl-lpi4a.bin
wget https://mirror.iscas.ac.cn/revyos/extra/images/lpi4a/20240720/root-lpi4a-20240720_171951.ext4.zst
zstd -d boot-lpi4a-20240720_171951.ext4.zst
zstd -d root-lpi4a-20240720_171951.ext4.zst
```

按住 boot 键后,上电/Reset 进入刷写模式。

刷写系统:
```shell!
sudo fastboot devices
sudo fastboot flash ram u-boot-with-spl-lpi4a.bin 
sudo fastboot reboot
sudo fastboot flash uboot u-boot-with-spl-lpi4a.bin
sudo fastboot flash boot boot-lpi4a-20240720_171951.ext4
sudo fastboot flash root root-lpi4a-20240720_171951.ext4
```
默认账号密码为:`debian`:`debian`

#### Milk-V Pioneer Box

- Fedora
从 [MilkV 官方](https://milkv.io/zh/docs/pioneer/getting-started/download) 给出的下载链接下载并解压后,使用 `dd` 刷入 **SD 卡**:
```shell!
sudo dd if=path/to/fedora-disk-gnome-workstation.raw of=/dev/sdx bs=1M status=progress
```
将 SD 卡插入 Pioneer Box 后面板的 SD 卡槽中后,启动系统。

下载 [`mv-rootfs.sh`](https://github.com/milkv-pioneer/scripts/blob/main/mv-rootfs.sh) 并以 `sudo` 在 Pioneer Box 上执行该脚本:
```shell!
wget https://github.com/milkv-pioneer/scripts/blob/main/mv-rootfs.sh
sudo bash ./mv-rootfs.sh
```
按提示选择硬盘,确认和自动在 NVMe 硬盘上安装。安装完成后移除 SD 卡并重启。

- openKylin
下载 [openKylin 2.0](https://mirrors.hust.edu.cn/openkylin-cdimage/2.0/openKylin-Embedded-V2.0-Release-milk-v-pioneer-riscv64.img.xz) 后,使用 `xz` 解压,`dd` 到 **NVMe 硬盘**中 *(需要一个硬盘盒)*:
```shell!
xz -kd openKylin-Embedded-V2.0-Release-milk-v-pioneer-riscv64.img.xz
sudo dd if=path/to/openKylin-Embedded-V2.0-Release-milk-v-pioneer-riscv64.img of=/dev/your/nvme bs=4M status=progress
```
将硬盘插入系统后自动启动。

- RevyOS
下载 [RevyOS 20241025](https://mirror.iscas.ac.cn/revyos/extra/images/sg2042/20241025/revyos-pioneer-20241025-001347.img.zst) 后,使用 `zstd` 解压,`dd` 到 **NVMe 硬盘**中 *(需要一个硬盘盒)*:
```shell!
zstd -d revyos-pioneer-20241025-001347.img.zst
sudo dd if=path/to/revyos-pioneer-20241025-001347.img of=/dev/your/nvme bs=4M status=progress
```
将硬盘插入系统后自动启动。

### 手动测试环境

- Fedora
使用 `dnf` 进行安装:
```shell!
sudo dnf update
sudo dnf install firefox
```
- openKylin
使用 `apt` 进行安装:
```shell!
sudo apt update
sudo apt upgrade
sudo apt install firefox
```
- RevyOS
使用 `apt` 进行安装:
```shell!
sudo apt update
sudo apt upgrade
sudo apt install firefox
```

### Firefox 自动测试环境

Firefox 自带的 wpt 自动测试需要构建出 Nightly 环境后进行使用;可参考 [官方文档](https://firefox-source-docs.mozilla.org/setup/linux_build.html)。

- 安装依赖:
```shell!
sudo dnf install python3 python3-pip
# apt based:
# sudo apt-get install curl python3 python3-pip
```

- 创建虚拟环境:
```shell!
python -m venv ~/venv_wpt
source ~/venv_wpt/bin/activate
```

- 安装 hg 工具:
```shell!
pip install mercurial
hg version
```

- 初始化 Firefox 源码:
*该步骤会下载超 10G 的源代码,请确保良好的网络连接*
```shell!
curl https://hg.mozilla.org/mozilla-central/raw-file/default/python/mozboot/bin/bootstrap.py -O

python bootstrap.py
```
选择 `2. Firefox for Desktop` 项目

- 配置依赖
Firefox 依赖于 Rust,需要采用 `rustup.sh` 安装 Rust 工具链:
```shell!
curl --proto '=https' --tlsv1.2 -sSf https://sh.rustup.rs | sh
```

其还依赖 clang 和 nodejs,根据环境安装:
```shell!
sudo dnf install clang clang-devel nodejs
# sudo apt-get install clang clang-dev nodejs
```

该部分由于环境不同,根据 configure 提示使用 apt/pip/cargo 补全依赖即可:
```shell!
./mach wpt configure
```
待到提示所有依赖成功安装后,可进行下一步构建。

- 构建 Nightly
使用 `mach` 脚本构建即可:
```shell!
./mach build
```
该过程可能耗时 2-3h 不等。

- 运行 wpt
**注意⚠️** 该测试完全运行极其缓慢,可能耗时超过 58 个小时(单 CPU 核心下)。并行测试由于未知原因无法进行,无法加速测试。
使用 `mach` 脚本运行 `wpt` 测试:
```shell!
./mach wpt run --log-raw wpt_out.txt --log-html wpt_out.html --log-mach wpt_out_readable.txt --log-wptreport wpt_out_report.txt --process=8
```

测试完成后,可在当前文件夹下找到测试结果。

### web-platform-tests 官方自动测试
直接使用 [web-platform-tests](https://web-platform-tests.org/) 中提供的测试。

- 安装环境
克隆其 git 仓库:
```shell!
git clone https://github.com/web-platform-tests/wpt.git
cd wpt
```

安装 Python 依赖:
```shell!
sudo apt-get install python3 python3-pip python3-venv python-is-python3
# sudo dnf install python3 python3-pip python3-venv python-unversioned-command
```

- 配置 hosts
向 hosts 中添加 wpt 的解析:
```shell!
./wpt make-hosts-file | sudo tee -a /etc/hosts
```

- 运行测试:
```shell!
./wpt --venv /home/openkylin/venv_wpt/ run --log-raw wpt_out.txt --log-html wpt_out.html --log-mach wpt_out_readable.txt --log-wptreport wpt_out_report.txt firefox -f --process 16
```
当提示是否安装 geckodriver 时,请选择否(官方无 RISC-V 版本),否则其会下载其它架构版本影响测试。

### 性能测试

- Speedometer 3

在 Firefox 浏览器中访问:https://browserbench.org/Speedometer3.0/

- Basemark

在 Firefox 浏览器中访问:https://web.basemark.com

# 测试内容

## 手动测试

手动测试中,我们模拟用户日常使用习惯精选了 72 个测试用例进行测试,涵盖了浏览网页、下载、书签、打印、设置、历史等常见操作,并以此基础进行了实际使用体验和步骤截图。

## web-platform-tests 自动测试

web-platform-tests(wpt)自动测试套件是一个跨浏览器的 Web 平台测试项目。其根据 Web 标准实现了一系列的测试用例,来确保浏览器可被认为是与 Web 兼容的。

整个测试套件涵盖了浏览器的几乎全部方面,从 DOM 节点、CSS 绘制、Cookies、交互等底层内容,到各个 API、外设使用、浏览器核心等均有涉及,组成了 57640 个测试,总计实现了超过两百万个子测试用例,可以认为是浏览器最权威的兼容性测试之一。

wpt 官方维护着一个 x86 架构的测试 [看板](https://wpt.fyi/results/?label=master&label=experimental&aligned),可以在上面找到单 wpt 的最新结果。

由于版权和证书问题,Firefox 自带的 wpt 套件中还包含了其自己转为 wpt 编写的专有用例,以`_mozilla`的路径显示,因此 Firefox 自身的报告测试数量会高于 wpt 官方测试的数目,为正常情况。

## 性能测试

- Speedometer 3

Speedometer 3 是 Web 浏览器的基准测试,它通过对各种工作负载上的模拟用户交互进行计时来测量 Web 应用程序响应能力。

- Basemark

Basemark Web 3.0 是一个全面的 Web 浏览器性能基准测试,可测试移动或桌面系统使用基于 Web 的应用程序的能力。该基准测试包括使用网络建议和功能的各种系统和图形测试。

# 测试结果

详细测试数据可见 [Github 仓库](https://github.com/wychlw/firefox_test/tree/main)

## 手动测试

手动测试 Firefox 可以正常使用。除解码可能由于没有外设支持导致无法硬解外,其余工作正常。

包括 Bing、Baidu、CSDN 和知乎等网站均能正常访问,账号功能正常可以登录和同步,书签、历史记录、下载等常用功能运行良好。

由于只有软解且 CPU 性能不足,视频播放遇到卡顿,在预期之中。

在 LicheePi 4A 上的 openKylin 系统上使用 Firefox 时会出现稳定性问题,以及无法解码播放在线视频(例如 Bilibili)。

## wpt 自动测试

**本节所述 Firefox 均为 Nightly 版本**

- Fedora
作为 Pioneer Box 官方支持的发行版,其兼容性较好。共 61486 个测试,通过 21487 个,跳过 216 个,预期错误 3409 个,已知问题 56 个;216 个失败,183 个错误,31 个预期无法通过的被通过。其余测试项无法在该平台上运行,或超时。

除 Firefox 在 CSS 上可能的固有差距外,报错大部分集中于 webgpu、webnn、webnfc 等需要与外设协同工作的组件。

- openKylin

其有一定兼容性,但和 Fedora 相比有一定差距。共 61486 个测试,通过 19663 个,跳过 479 个,预期错误 3414 个,已知问题 62 个;2036 个失败,142 个错误,32 个预期无法通过的被通过。其余测试项无法在该平台上运行,或超时。

除上述 Fedora 上出现的问题外,openKylin 上在 HTML Canvas、DOM 绘制、渲染方面也出现了一些问题。这可能与系统外设驱动、渲染驱动不同,或字体渲染错误等原因造成。

根据测试截图,差异一般较为细微,如绘图偏移、加粗、字体排版差异等,我们认为一般不影响日常使用。

## 性能测试

- Speedometer 3

Pioneer Box 得分为 `0.747` 分,LicheePi 4A 得分为 `0.3249`分。作为参考,现代 x86 计算机得分一般为 10-20 分。

- Basemark

Pioneer Box 最终得分为 `39.02` 分。作为参考,配备了高性能独立显卡(AMD RX 6600)的现代 x86 计算机的得分为 2000 分左右。

LicheePi 4A 不支持测试所需图形渲染。

Pioneer Box 的 Basemark Web 测试兼容性数据:
- - CSS Capabilities :59.18%
- - HTML5 Capabilities :91.35%
- - Page Load and Responsiveness Capabilities :91.59%
- - Resize Capabilities :75.86%

# 总结

综上所述,我们针对 Firefox 在 RISC-V 平台上的可用性进行了多项手动和自动测试。截止到 2024 年 11 月,Firefox 在 Milk-V Pioneer Box 和 Sipeed Lichee Pi 4A 上运行较为良好。受限于 SG2042 和 TH1520 的单核性能,并不能做到非常流畅的运行。也有诸如无法使用硬件解码播放视频等问题。Firefox 在 RISC-V 平台上的支持仍然任重而道远。


\end{markdown}

\newpage
\section*{附录}

\appendix

\reference

\end{document}
